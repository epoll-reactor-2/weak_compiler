\section{Optymalizacje (implementacja)}
	\subsection{Unreachable code elimination}
	
		Usuwanie kodu nieosiągalnego polega na dwóćh krokach.
		\begin{itemize}
			\item Obejście grafu sterowania programu (\textbf{CFG})
			\item Usuwanie wszystkich instrukcji, do których nie ma żadnych wejściowych krawędzi.
		\end{itemize}
		
		\spacing

		Algorytm polega na zlalezieniu takich \textbf{podgrafów} grafu sterowania programem, do których
		nie prowadzi żadna z krawędzi.
		\\

		\begin{algorithm}
			\caption{Usuwanie kodu nieosiągalnego}
			\begin{algorithmic}[1]
				\Procedure{Eliminate}{CFG}
					\State Visited $\gets$ $\emptyset$
					\State \Call{Traverse}{Visited, CFG, First(CFG)}
					\State Unvisited $\gets$ CFG $\setminus$ Visited
					\State \Call{Cut}{Unvisited, CFG}
				\EndProcedure
				\\
				\Procedure{Traverse}{Visited, CFG, IR}
					\State Visited[IR] $\gets$ 1
					\For{each control flow successor of IR}
						\State \Call{Traverse}{Visited, CFG, Succ(IR)}
					\EndFor
				\EndProcedure
				\\
				\Procedure{Cut}{Unvisited, CFG}
					\For{each unvisited statement}
						\State Remove statement from IR
					\EndFor
				\EndProcedure
			\end{algorithmic}
		\end{algorithm}
