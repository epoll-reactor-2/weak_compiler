\section{Analiza leksykalna}

	Jednym ze sposobów na sprowadzanie kodu źródłowego do postaći listy tokenów jest narzędzie flex.
	Przyjmuje ono zestaw reguł w postaći wyrażeń regularnych, według których
	działa rozbicie tekstu wejściowego. Można jednak ominąć lex i zaimplemenetować lexer ręcznie, ale
	ta praca nie skupia się na tym.

	\subsection{Wyrażenia regularne}

		Wyrażeniem regularnym nad alfabetem $\sum$ nazywamy ciąg znaków
		$\emptyset, \epsilon, +, *, ), ($ wraz z alfabetem $\sum$. Symbole $a_i$ z alfabetu $\sum$ mają
		postać:
		
		\begin{itemize}
		    \item $\emptyset, \epsilon$ są wyrażeniami regularnymi.
		    \item Wszystkie $a_i \in \sum$ są wyrażeniami regularnymi.
		    \item Jeśli $a_1$, $a_2$ są wyrażeniami regularnymi, to są nimi również
		    \begin{itemize}
		        \item $a_1^*$ -- domknięcie Kleene'go.
		        \item $a_1a_2$ -- konkatenacja.
		        \item $a_1 + a_2$ -- suma.
		        \item $(a_1)$ -- grupowanie.
		    \end{itemize}
		\end{itemize}
		
		Każde wyrażenie regularne definiuje się przez automat skończony deterministyczny. \\
		Każde wyrażenie regularne \textbf{akceptuje} pewnien język formalny. Pod akcętpacją
		mamy na uwadze sekwencję przejść pomiędzy stanami automatu skończonego, która prowadzi
		do stanu końcowego. \\
		
		Zauważmy, że w praktyce stosuje się nieco innych, poszerzonych wyrażen regularnych. Ich notacja różni się
		od notacji matematycznej. Dalej omówione są wyrażenia regularne, zgodne z POSIX. \\

		Podamy przykład automatu dla wyrażenia \texttt{-?[0-9]+}
		
		\spacing
		\spacing

		\begin{center}
		\begin{tikzpicture}[node distance=4cm]
			\node[state, initial, accepting] (0) {0};
			\node[state, right of=0]         (1) {1};
			\node[state, right of=1]         (2) {2};
			\draw
			(0) edge[       bend left ]             (1)
			(0) edge[below, bend right] node{$-$}   (1)
			(1) edge[below            ] node{$0-9$} (2)
			(2) edge[loop below       ] node{$0-9$} (2)
			;
		\end{tikzpicture}
		\end{center}
		
		Aby odśledzić wykonane kroki, można wypełnić tabelę przejść pomiędzy stanami. Podamy przykład dla
		łańcucha \texttt{-22}

		\spacing
		\spacing

	\begin{center}
		\setlength{\tabcolsep}{0.5em}
		\renewcommand{\arraystretch}{1.2}
		\begin{tabular}{ | L{3cm} | L{4cm} | }
			\hline
			Biężący stan        & Akcja \\
			\hline
			\textbf{0}          & zaakceptować - \\
			\hline
			0, \textbf{1}       & zaakceptować 2 \\
			\hline
			0, 1, \textbf{2}    & zaakceptować 2 \\
			\hline
		\end{tabular}
	\end{center}
	
	\newpage

\subsection{Flex}

	Flex jest narzędziem projektu GNU. Pozwala ono w wygodny sposób podać listę reguł dla analizy
	leksykalnej (ang. Scanning). Flex jest mocno powiązany z językiem C, dlatego program w flex'u
	korzysta z konstrukcji języka C. Pokażemy przykład użycia flex'u
	
	\spacing
	\lstinputlisting[label={lst:ast-clang}]{code/flex.txt}
	\spacing
	
	Zauważmy, że flex próbuje szukać wzorców w tekscie dokładnie w takiej kolejności, która jest podana w
	jego kodzie. Dlatego często robią ostatnią regułe z wyrażeniem regularnym ".", który akceptuje
	dowolny znak, i umieszczają tam komunikat o błędzie.
	\\

	W naszym przypadku, lex generuje kod, który gromadzi wszystkie znalezione lexemy do tablicy.

	\newpage
