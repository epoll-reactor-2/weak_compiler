\section{Historia}
	
	Potrzeba automatyzacji pracy intelektualnej istniała zawsze. Dlatego od dawna człowiek próbuje
	znaleźć metody do tego. Niżej jest krótkie podsumowanie powstania informatyki.
	\\

	\begin{itemize}
		\item W \textbf{IX} wieku
			przez irańskiego matematyka al Kindi wieku został stworzony system szyfrowania informacji na
			podstawie zliczania ilości liter w tekscie.

		\item W \textbf{XVII} wieku powstał suwak logarytmiczny, potrzebny do ułatwienia działań
			matematycznych.
		
		\item W tym samym \textbf{XVII} wieku powstał jeden z pierszych kalkulatorów mechanicznych
			\textbf{Pascalina}. Jest to narzędzie do wykonania operacji arytmetycznych na podstawie
			ruchu koł zębatych i innych części.
			
		\item W \textbf{XVIII} wieku Charlesa Babbage stworzył mechaniczną \textbf{maszynę
			różnicową} do tworzenia dużych tabeli logarytmicznych, które do tej pory człowiek
			musiał wyliczać ręcznie.
		
		\item W 1847 roku George Boole wyprowadził nowy rozdział algebry: \textbf{algebrę Boole'a},
			na podstawie której później został zaprojektowany pierwszy klasyczny komputer.
		
		\item W 1930 roku Vannevar Bush stworzył \textbf{analizator różnicowy} do rozwiązania
			równań różnicowych metodą całkowania.
	\end{itemize}