\section{Analiza semantyczna}

	Aby zapewnić poprawność napisanego kodu, stosuje się wiele rodzajów analiz. Niniejszy kompilator
	dysponuje trzema:
		
	\begin{itemize}
		\item Analiza nieużytych zmiennych, oraz zmiennych, które są zdefiniowane, ale nie zostały użyte
		\item Analiza poprawności typów
		\item Analiza prawidłowego użycia funkcji
	\end{itemize}
	
	\subsection{Analiza nieużywanych zmiennych}
		
		Podamy przykłady kodu prowadzącego do odpowiednich ostrzeżeń
		
		\spacing

\begin{lstlisting}[label={lst:warn-unused-var}]
void f() {
	int argument = 0; // Warning: unused variable `argument`
}
\end{lstlisting}

\begin{lstlisting}[label={lst:warn-unused-var}]
void f(int argument) { // Warning: unused variable `argument`

}
\end{lstlisting}
			
\begin{lstlisting}[label={lst:warn-unused-var}]
void f() {
	int argument = 0;
	++argument; // Warning: variable `argument` written, but never read
}
\end{lstlisting}

	\spacing
	
	Rzecz polega na przejściu drzewa syntaksycznego i zwiększania liczników
	\textbf{read_uses} i \textbf{write_uses} dla każdego węzła typu
	\textbf{ast_sym}.
	\\
	
	Algorytm operuje na blokach kodu, zawartego w \textbf{\{ ... \}}. Po przejściu każdego bloku
	(w tym rekurencyjnie), analiza jest wykonana w następujący sposób:

	\begin{algorithm}
		\caption{Wyszukiwanie nieużywanych zmiennych}
		\begin{algorithmic}[1]

		\Procedure{Analyze}{AST}
			\State Set $\gets$ all declarations at current scope depth

			\For{each collected declaration Use in Set}
				\If{Use is not a function \& Use.ReadUses is 0}
					\State Emit warning
				\EndIf
			\EndFor
		\EndProcedure

		\end{algorithmic}
	\end{algorithm}

	Do analizy nieużywanych funkcji stosuje się tem sam algorytm. Jedyne, co jest wtedy zmienione --
	sprawdzenie, czy nazwa rozpatrywanej funkcji nie jest \textbf{main}. Funkcja \textbf{main} jest
	wywołana automatycznie.
