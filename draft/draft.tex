\documentclass[leqno]{article}

\usepackage[fleqn]{amsmath}  % Aligned to left equations.
\usepackage{syntax}
\usepackage{fancyhdr}
\usepackage{array}

\newcolumntype{L}[1]{>{\raggedright\let\newline\\\arraybackslash\hspace{0pt}}m{#1}}
\newcolumntype{C}[1]{>{\centering\let\newline\\\arraybackslash\hspace{0pt}}m{#1}}
\newcolumntype{R}[1]{>{\raggedleft\let\newline\\\arraybackslash\hspace{0pt}}m{#1}}

\title{Weak language draft}
\author{epoll-reactor}
\date{since December 2021}

\begin{document}
    \pagestyle{fancy}
    \rhead{\leftmark}
    \lhead{Weak language}

    \maketitle
    \tableofcontents

    \newpage

    \section{Scope}
        This document describes requirements for implementation of weak
        programming language. \\
        \\
        \textbf{\thesubsection.1} Language is not required to have one implementation, \\
        \textbf{\thesubsection.2} as well as the implementation is not required to be
        a compiler.

    \section{Lexical elements}
        \upshape

        \subsection{Keywords}
            \begin{tabular}{ R{3cm} R{3cm} R{3cm} }
                \tt{boolean}  & \tt{break}  & \tt{char}   \\
                \tt{continue} & \tt{do}     & \tt{else}   \\
                \tt{false}    & \tt{float}  & \tt{for}    \\
                \tt{if}       & \tt{int}    & \tt{return} \\
                \tt{string}   & \tt{struct} & \tt{true}   \\
                \tt{void}     & \tt{while}
            \end{tabular}

        \subsection{Operators and punctuators}
            \begin{tabular}{ R{1.5cm} R{1.5cm} R{1.5cm} R{1.5cm} R{1.5cm} R{1.5cm} R{1.5cm} }
                \tt{=}   & \tt{*=}     & \tt{/=}  & \tt{\%=} & \tt{+=}   & \tt{-=}   \\
                \tt{<<=} & \tt{>>=}    & \tt{\&=} & \tt{|=}  & \tt{\^=}  & \tt{\&\&} \\
                \tt{||}  & \tt{\^=}    & \tt{\&}  & \tt{|}   & \tt{==}   & \tt{!=}   \\
                \tt{>}   & \tt{<}      & \tt{>}   & \tt{<=}  & \tt{>=}   & \tt{<<}   \\
                \tt{>>}  & \tt{+}      & \tt{-}   & \tt{*}   & \tt{/}    & \tt{\%}   \\
                \tt{++}  & \tt{--}     & \tt{.}   & \tt{,}   & \tt{;}    & \tt{!}    \\
                \tt{[}   & \tt{]}      & \tt{(}   & \tt{)}   & \tt{\{}   & \tt{\}}   \\
            \end{tabular}

        \subsection{Comments}
            Comments are not involved into the parsing and should be processed at the
            lexical analysis stage. \\
            \\
            \textbf{\thesubsection.1} All text starting with // should be ignored until the end of line. \\
            \textbf{\thesubsection.2} All text after /*  and before */ character sequences should be ignored.

    \section{Data types}
        \upshape
            \textbf{\thesubsection.1} Language must implement static strong type system. \\
            \textbf{\thesubsection.2} All casts must be explicit.

            \subsection{Primitive}
                \textbf{int} -- signed 32-bit; \\
                \textbf{float} -- signed 32-bit; \\
                \textbf{char} -- 8-bit; \\
                \textbf{bool} -- 8-bit; \\
                \textbf{void} -- empty type; used as return type only. \\
                \\
                \textbf{\thesubsection.1} Primitive types must be trivial copyable. It means
                \textbf{memcpy()} must copy all primtive types between memory locations
                without side effects. \\
                \textbf{\thesubsection.2} No one primitive type can be converted to another implicitly.
                \subsubsection{String}
                
                \textbf{\thesubsubsection.1} Character sequences must be represented
                as arrays of character type, for example,
                \begin{align*}
                    \tt{char\ string[25];}
                \end{align*}
                or
                \begin{align*}
                    \tt{char\ *string = /* Implementation-defined. */;}
                \end{align*}
                , where implementation-defined part is some memory allocation.

            \subsection{Aggregate}
                \textbf{struct} -- aggregate type; may constructed from primitive
                types and from structures. \\
                \\
                \textbf{\thesubsection.1} Structures may be nested. \\
                \textbf{\thesubsection.2} Maximum structures depth is implementation-defined. \\
                \textbf{\thesubsection.3} Structure may not contain fields (be empty).

            \subsection{Arrays}
                Each type except \textbf{void} can represent array, for example,
                \begin{equation} \tag{\textbf{\thesubsection.1}}
                    \tt{bool\ array[10];}
                \end{equation}
                \textbf{\thesubsection.2} Arrays should (but not required to) be zero-initialized. \\
                \textbf{\thesubsection.3} Arrays can be multi-dimensional with arity of any depth
                greather than 0. \\
                \textbf{\thesubsection.4} Minimum array size must be at least 1 byte. \\
                \textbf{\thesubsection.5} Maximum array size is implementation-defined. \\
                \textbf{\thesubsection.6} Array indexing must start from 0. Zero index must
                point to first array element, and so on.

            \subsection{Pointers}
                \begin{equation} \tag{\textbf{\thesubsection.1}}
                    \tt{int\ *ptr\ =\ ...;}
                \end{equation}
                Address of the variable can be taken with \textbf{\&} operator, for example
                \begin{equation} \tag{\textbf{\thesubsection.2}}
                    \tt{int\ *ptr\ =\ \&var;}
                \end{equation}
                \textbf{\thesubsection.3} Pointer cannot have \textbf{NULL} (\textbf{0}) value. \\
                \textbf{\thesubsection.4} Pointers cannot be converted to another pointer type in any way. \\
                \textbf{\thesubsection.5} Pointer to \textbf{void} is forbidden. All other types (including structures) are allowed to have pointer to it. \\
                \textbf{\thesubsection.6} Pointers must allow to change the same memory
                location through them (for example, in other function).
                \begin{align*}
                    \tt{void\ f(int\ *ptr)\ \{\ *ptr\ =\ 100;\ \} }
                \end{align*}
                Now \textbf{ptr} must be equal 100 outside this context. \\
                \textbf{\thesubsection.7} \textbf{\&} operator may be applied to
                non-pointer type.
                \begin{align*}
                    \tt{int\ mem\ =\ ...;\ int\ *ptr\ =\ \&mem;}
                \end{align*}
                \textbf{\thesubsection.8} \textbf{\&} operator may be applied to pointer
                type.
                \begin{align*}
                    \tt{int\ *ptr1\ =\ ...;\ int\ **ptr2\ =\ \&ptr1;}
                \end{align*}

    \section{Inside-iteration statements}

        \subsection{Break}
            \begin{equation} \tag{\textbf{\thesubsection.1}}
                \tt{break;}
            \end{equation}
            \textbf{\thesubsection.2} \textbf{break} usable only inside \textbf{while},
            \textbf{do-while} and \textbf{for} statements. \\
            \textbf{\thesubsection.3} \textbf{break} performs exit from a loop.

        \subsection{Continue}
            \begin{equation} \tag{\textbf{\thesubsection.1}}
                \tt{continue;}
            \end{equation}
            \textbf{\thesubsection.2} \textbf{continue} usable only inside the \textbf{while},
            \textbf{do-while} and \textbf{for} statements. \\
            \textbf{\thesubsection.3} \textbf{continue} performs jump to the next loop iteration.

    \section{Iteration statements}

        \subsection{While}
            \begin{equation} \tag{\textbf{\thesubsection.1}}
                \tt{while\ (<stmt>)\ \{\ <stmt>*\ \}}
            \end{equation}
            \textbf{\thesubsection.2} \textbf{while} is the loop statement that executes its
            body until the condition evaluates to false.

        \subsection{Do-While}
            \begin{equation} \tag{\textbf{\thesubsection.1}}
                \tt{do\ \{\ <stmt>*\ \} \ while\ (<stmt>);}
            \end{equation}
            \textbf{\thesubsection.2} \textbf{do-while} is the loop statement with similar to 
            \textbf{while} semantics, but it executes body before contition check at first time.

        \subsection{For}
            \begin{equation} \tag{\textbf{\thesubsection.1}}
                \tt{for\ (<stmt>\ |\ \epsilon;\ <stmt>\ |\ \epsilon;\ <stmt>\ |\ \epsilon)\ \{\ <stmt>*\ \}}
            \end{equation}
            \textbf{\thesubsection.2} \textbf{for} is the loop statement with three initial parts
            and body. \\
            \\
            \textbf{\thesubsection.3} Initial part \\
            \textbf{\thesubsection.3.1} should (but not required to) have variable
            assignment; \\
            \textbf{\thesubsection.3.2} executed once before all remaining statements in
            \textbf{for} loop. \\
            \\
            \textbf{\thesubsection.4} Conditional part \\
            \textbf{\thesubsection.4.1} should (but not required to) have expression convertible to boolean; \\
            \textbf{\thesubsection.4.2} executed before each body part execution. \\
            \\
            \textbf{\thesubsection.5} Incremental part \\
            \textbf{\thesubsection.5.1} should (but not required to) have binary or unary expression.

    \section{Conditional statements}
        \subsection{If}
            \begin{equation} \tag{\textbf{\thesubsection.1}}
                \tt{if\ ( <stmt> )\ \{ <stmt>* \}}
            \end{equation}
            \begin{equation} \tag{\textbf{\thesubsection.2}}
                \tt{if\ ( <stmt> )\ \{ <stmt>* \}\ else\ \{ <stmt>* \}}
            \end{equation}
            \textbf{\thesubsection.3} \textbf{if} is the statament that executes its body if condition is true. \\
            \textbf{\thesubsection.4} \textbf{if} statement optionally can have \textbf{else} part. \\
            \textbf{\thesubsection.5} if condition is false and \textbf{else} part exists, it
            should be executed.

    \section{Jump statements}
        \subsection{Return}
            \begin{equation} \tag{\textbf{\thesubsection.1}}
                \tt{return;}
            \end{equation}
            \begin{equation} \tag{\textbf{\thesubsection.2}}
                \tt{return <stmt>;}
            \end{equation}
            \textbf{\thesubsection.3} \textbf{return} is the statement being the end of control flow. \\
            \textbf{\thesubsection.4} If parent function have return type other than
            \textbf{void}, value of this type must be returned on every exit point. \\
            \textbf{\thesubsection.5} If parent function have \textbf{void} return type,
            control flow may (but not required to) break at any point whith \textbf{return;}
            statement.
    
    \section{Functions}
        \begin{equation} \tag{\textbf{\thesubsection.1}}
            \tt{<ret-type>}\ \tt{<id>}\ \tt{(}\ \tt{<parameter-list>}\ \tt{);}
        \end{equation}
        \begin{equation} \tag{\textbf{\thesubsection.2}}
            \tt{<ret-type>}\ \tt{<id>}\ \tt{(}\ \tt{<parameter-list>}\ \tt{)}\ \tt{\{}\ <stmt>*\ \tt{\}}
        \end{equation}

        \subsection{Return value}
            Function can return \\
            \textbf{\thesubsection.1} any primitive type, or \\
            \textbf{\thesubsection.2} pointer to any primitive type, or \\
            \textbf{\thesubsection.3} structure type, or \\
            \textbf{\thesubsection.4} pointer to structure type, or \\
            \textbf{\thesubsection.5} nothing (\textbf{void}).\\
            \\
            \textbf{\thesubsection.6} Return value can be ignored (if any).
        
        \subsection{Recursion}
            \textbf{\thesubsection.1} Function can call itself (recursive call). \\
            \textbf{\thesubsection.2} Maximum recursive call depth is implementation-defined.

        \subsection{Arguments}
            \textbf{\thesubsection.1} Arguments count must be at least 0. \\
            \textbf{\thesubsection.2} Maximum arguments count is implementation-defined. \\
            \textbf{\thesubsection.3} To be called, a function call statement must have
            the same arguments count as its declaration. \\
            \textbf{\thesubsection.4} To be called, a function call statement must have
            the same argument types as its declaration. \\
            \textbf{\thesubsection.4} Pointers can be passed to function call statement.

    \section{Execution flow}
        \textbf{\thesubsection.1} Code can be executed only inside functions. \\
        \textbf{\thesubsection.2} Entry point is the function
            \begin{align*}
                \tt{int\ main()\ \{\ ...\ \}}
            \end{align*}
            or
            \begin{align*}
                \tt{int\ main(int\ argc, char **argv)\ \{\ ...\ \}}
            \end{align*}
        \textbf{\thesubsection.3} Main function can optionally support command-line
        arguments. \\
        \textbf{\thesubsection.4} \textbf{argc} must be 1 or greater. \\
        \textbf{\thesubsection.4} \textbf{argv[0]} must contain executable file name or any representation of program file (in case if this is an interpreter). \\
        \textbf{\thesubsection.5} Command-line arguments are passed to main function
        through received from operation system values.

    \section{FFI}
        \subsection{Call conventions}
            \textbf{\thesubsection.1} \textbf{cdecl} call convention must be used in order
            to be able to link with other shared libraries supporting this convention.

    \section{Grammar summary}
        \itshape
        \setlength{\grammarindent}{12em}

        \begin{grammar}
            <program> ::= ( <function-decl> | <structure-decl> )*

            <structure-decl> ::= \tt{struct} \tt{\{} <structure-decl-list> \tt{\}}

            <structure-decl-list> ::= (
            <decl-without-initialiser> \tt{;}
            \alt <structure-decl> \tt{;}
            )*

            <function-decl> ::= <ret-type> <id> \tt{(} <parameter-list-opt> \tt{)} \tt{\{} <stmt>* \tt{\}}

            <ret-type> ::= <type>
            \alt <void-type>

            <type> ::= \tt{int}
            \alt \tt{float}
            \alt \tt{char}
            \alt \tt{string}
            \alt \tt{boolean}

            <void-type> ::= \tt{void}

            <constant> ::= <integral-literal>
            \alt <floating-literal>
            \alt <string-literal>
            \alt <char-literal>
            \alt <boolean-literal>

            <integral-literal> ::= <digit>*
            
            <floating-literal> ::= <digit>* \tt{.} <digit>*

            <string-literal> ::= ''\tt{( "x00000000-x0010FFFF" )*}''

            <char-literal> ::= '\tt{ASCII(0)}-\tt{ASCII(127)}'

            <boolean-literal> ::= \tt{true}
            \alt \tt{false}

            <alpha> ::= \tt{a} | \tt{b} | ... | \tt{z} | \tt{_}

            <digit> ::= \tt{0} | \tt{1} | ... | \tt{9}

            <id> ::= <alpha> ( <alpha> | <digit> )*

            <array-decl> ::= <type> ( \tt{*} )* <id> \tt{[} <integral-literal> \tt{]}

            <var-decl> ::= <type> ( \tt{*} )* <id> \tt{=} <logical-or-stmt> \tt{;}
            
            <structure-var-decl> ::= <id> ( \tt{*} )* <id>

            <decl> ::= <var-decl>
            \alt <array-decl>
            \alt <structure-var-decl>

            <decl-without-initialiser> ::= <type> ( \tt{*} )* <id>
            \alt <array-decl>
            \alt <structure-var-decl>

            <parameter-list> ::= <decl-without-initialiser> \tt{,} <parameter-list>
            \alt <decl-without-initialiser>

            <parameter-list-opt> ::= <parameter-list> | $\epsilon$

            <stmt> ::= <block-stmt>
            \alt <selection-stmt>
            \alt <iteration-stmt>
            \alt <jump-stmt>
            \alt <decl>
            \alt <expr>
            \alt <assignment-stmt>
            \alt <primary-stmt>
            
            <member-access-stmt> ::= <id> \tt{.} <member-access-stmt>
            \alt <id> \tt{.} <id>

            <iteration-stmt> ::= <stmt>
            \alt \tt{break};
            \alt \tt{continue};
            
            <block-stmt> ::= \tt{\{} <stmt>* \tt{\}}
            
            <iteration-block-stmt> ::= \tt{\{} <iteration-stmt>* \tt{\}}

            <selection-stmt> ::= \tt{if} \tt{(} <expr> \tt{)} <block-stmt>
            \alt \tt{if} \tt{(} <expr> \tt{)} <block-stmt> \tt{else} <block-stmt>

            <iteration-stmt> ::= \tt{for} \tt{(} <expr-opt> \tt{;} <expr-opt> \tt{;} <expr-opt> \tt{)} <iteration-block-stmt>
            \alt \tt{while} \tt{(} <expr> \tt{)} <iteration-block-stmt>
            \alt \tt{do} <iteration-block-stmt> \tt{while} \tt{(} <expr> \tt{)} \tt{;}

            <jump-stmt> ::= \tt{return} <expr> ? \tt{;}

            <assignment-op> ::= \tt{=}
            \alt \tt{*=}
            \alt \tt{/=}
            \alt \tt{\%=}
            \alt \tt{+=}
            \alt \tt{-=}
            \alt \tt{"<<="}
            \alt \tt{">>="}
            \alt \tt{\&=}
            \alt \tt{|=}
            \alt \tt{"^="}

            <expr> ::= <assignment-stmt>
            \alt <var-decl>

            <expr-opt> ::= <expr> | $\epsilon$

            <assignment-stmt> ::= <logical-or-stmt>
            \alt <logical-or-stmt> <assignment-op> <assignment-stmt>

            <logical-or-stmt> ::= <logical-and-stmt>
            \alt <logical-and-stmt> \tt{||} <logical-or-stmt>

            <logical-and-stmt> ::= <inclusive-or-stmt>
            \alt <inclusive-or-stmt> \tt{\&\&} <logical-and-stmt>

            <inclusive-or-stmt> ::= <exclusive-or-stmt>
            \alt <exclusive-or-stmt> \tt{|} <inclusive-or-stmt>

            <exclusive-or-stmt> ::= <and-stmt>
            \alt <and-stmt> \tt{"^"} <exclusive-or-stmt>

            <and-stmt> ::= <equality-stmt>
            \alt <equality-stmt> \tt{\&} <and-stmt>

            <equality-stmt> ::= <relational-stmt>
            \alt <relational-stmt> \tt{==} <equality-stmt>
            \alt <relational-stmt> \tt{!=} <equality-stmt>

            <relational-stmt> ::= <shift-stmt>
            \alt <shift-stmt> \tt{">"} <relational-stmt>
            \alt <shift-stmt> \tt{"<"} <relational-stmt>
            \alt <shift-stmt> \tt{">="} <relational-stmt>
            \alt <shift-stmt> \tt{"<="} <relational-stmt>

            <shift-stmt> ::= <additive-stmt>
            \alt <additive-stmt> \tt{"<<"} <shift-stmt>
            \alt <additive-stmt> \tt{">>"} <shift-stmt>

            <additive-stmt> ::= <multiplicative-stmt>
            \alt <multiplicative-stmt> \tt{"+"} <additive-stmt>
            \alt <multiplicative-stmt> \tt{"-"} <additive-stmt>

            <multiplicative-stmt> ::= <prefix-unary-stmt>
            \alt <prefix-unary-stmt> \tt{"*"} <multiplicative-stmt>
            \alt <prefix-unary-stmt> \tt{"/"} <multiplicative-stmt>
            \alt <prefix-unary-stmt> \tt{"%"} <multiplicative-stmt>

            <prefix-unary-stmt> ::= <postfix-unary-stmt>
            \alt \tt{++} <postfix-unary-stmt>
            \alt \tt{--} <postfix-unary-stmt>
            \alt \tt{*} <postfix-unary-stmt>
            \alt \tt{\&} <postfix-unary-stmt>
            \alt \tt{!} <postfix-unary-stmt>

            <postfix-unary-stmt> ::= <primary-stmt>
            \alt <primary-stmt> \tt{++}
            \alt <primary-stmt> \tt{--}

            <primary-stmt> ::= <constant>
            \alt <symbol-stmt>
            \alt \tt{(} <logical-or-stmt> \tt{)}
            
            <symbol-stmt> ::= <function-call-stmt>
            \alt <array-access-stmt>
            \alt <member-access-stmt>
            \alt <id>
            
            <array-access-stmt> ::= <id> ( \tt{[} <expr> ] )*
            
            <function-call-arg-list> ::= <logical-or-stmt> \tt{,} <function-call-arg-list>
            \alt <logical-or-stmt>
            
            <function-call-arg-list-opt> ::= <function-call-arg-list> | $\epsilon$

            <function-call-expr> ::= <id> \tt{(} <function-call-arg-list-opt> \tt{)}

        \end{grammar}

\end{document}